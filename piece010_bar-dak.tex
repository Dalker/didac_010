\documentclass[11pt,bibliography=totoc]{scrartcl}

% Son titrage
\author{Juan-Carlos Barros et Daniel Kessler}
\date{\today}
\title{Conception d'une séquence d'enseignement\\\medskip
  \large GymInf, Didactique de l'informatique, Rendu 010}

% son français
\usepackage[french]{babel}
\usepackage{csquotes}

% Sa gestion de bibliographie
% NB: nos consignes nous demandent d'utiliser les normes APA
%     ça tombe bien, c'est possible de le préciser ici!
\usepackage[backend=biber,style=apa,
            urldate=edtf,date=edtf,seconds=true]{biblatex}
\addbibresource{biblio.bib}

% configuration générale
\usepackage[l2tabu, orthodox]{nag}  % demander indication d'usages désuets
\usepackage{microtype}  % subtiles améliorations de certains défauts
\usepackage[french]{babel}  % langue du document
\usepackage{graphicx}  % pour inclure des images et des pdf

% configuration des polices de caractère / aspect du titre
\usepackage[utf8]{inputenc}  % nécessaire en MS-Windows, pas dans Linux
                             % (probablement pas dans MacOS non plus)
% Possibles changements de police ci-dessous (en décomenter un, ou pas)
\usepackage[T1]{fontenc}  % <- c'est nécessaire dans window$, je crois
% \usepackage{tgtermes} %  (semble nécessiter lualatex)

% configuration géométrie / en-tête / pieds de page
% remplacer 2.2 par 1.2 ci-dessous avec scrartcl (mise en page de base différente)
\addtolength{\voffset}{-2.2cm}   % reprendre la place de l'en-tête inexistante
\addtolength{\textheight}{4.2cm} % laisser plus de place pour le pied de page
\usepackage[compatV3]{fancyhdr}  % gestion d'en-tête et pied de page
\pagestyle{fancy}
\renewcommand\headrulewidth{0pt}   % pas de barre sous en-tête
\renewcommand\footrulewidth{0.4pt} % barre sur pied de page
\usepackage{lastpage}
\lfoot{pièce 010} \cfoot{} \rfoot{page \thepage/\pageref{LastPage}}
\fancypagestyle{plain}{  % style pour page de titre (pas de bas de page)
  \fancyhf{}\fancyfoot{}\renewcommand\footrulewidth{0pt} % barre sur pied de page
}
% pdfisation avec liens un peu bleus mais pas trop
\usepackage[pdfstartview=FitH,
pdfauthor={Juan-Carlos Barros et Daniel Kessler},
pdftitle={Conception d'une séquence}]{hyperref}
\usepackage{xcolor}
\definecolor{pdflinkcolor}{RGB}{30,60, 120}
\hypersetup{colorlinks=true, allcolors=pdflinkcolor}

\begin{document}
\thispagestyle{plain}  % page de garde sans bas de page

% Démarrage avec titre et table des matières sur page séparée
\maketitle
\tableofcontents  % <- au début, on n'est pas des frouzes qui mettent ça à la fin...
\pagebreak

% Et c'est parti pour du contenu!
\section{Contexte institutionnel et disciplinaire}
Nous enseignons tous deux au niveau gymnasial dans le Collège de Genève. Dans ce
cadre, nous avons cette année plusieurs groupes d'élèves de 1ère année
informatique discipline fondamentale, auxquels s'adresse le travail présenté
dans ce document. Ces groupes sont constitués de 16 élèves au maximum, issus
potentiellement de plusieurs ``classes'' (le concept de ``classe'' est très flou
au Collège, les élèves changeant souvent de groupe pour des matières à effectif
réduit, selon leur niveau de maths, leurs choix d'options, etc.).

La séquence d'enseignement que nous allons présenter possède deux
caractéristiques distinctives:
\begin{itemize}
\item Elle se déroule sur plusieurs leçons, en prenant 20-30 minutes du cours à
  chaque fois.
\item Elle combine des objectifs explicites pour les élèves avec des objectifs
  implicites qui seront explicités plus tard.
\end{itemize}

Cette séquence s'insère en deuxième partie du premier semestre, après avoir
travaillé sur le code binaire, vu quelques encodages simples (nombres entiers,
caractères \textsc{ascii}, bitmaps) et travaillé un petit peu sur les composants
matériels d'un ordinateur, éléments du chapitre \textit{Information et Données}
du plan d'étude cantonal genevois \autocite{pecinfo}.

L'objectif explicite initial de la séquence présentée dans ce document est
d'approfondir un des éléments précédents, à travers un travail de groupe sur un
sujet qui n'a été qu'effleuré pendant la première partie de l'année scolaire
(par exemple l'encodage des nombres à virgule flottante, la parenté entre les
composants d'un smartphone et ceux d'un PC, etc.)

Le deuxième objectif, initialement implicite, est de comprendre comment
fonctionne un \textit{wiki} en en utilisant un comme outil, pour découvrir
finalement que certains wikis sont très ouverts aux contributions publiques,
notamment \textit{Wikipedia}. Après avoir expérimenté le travail sur un wiki, un
élève devrait comprendre le mécanisme de base et se sentir capable de contribuer
à Wikipedia s'il ou elle y repère une coquille à corriger, une information
caduque ou incomplète, etc. Ce deuxième objectif participe à plusieurs thèmes du
plan d'étude notamment les \textit{Réseaux}, \textit{Information et Données} et
la \textit{Citoyenneté Numérique}.

\section{Objectifs pédagogiques}
La majorité de cette séquence se déroule en suivant le paradigme
\textit{socio-constructiviste} (cf. \cite{kozanitis}). En effet, les élèves
construisent une partie de leur propre connaissance en effectuant leurs propres
recherches et en collaborant avec leurs camarades.

Cependant, cela n'empêche pas quelques moments ``frontaux'', notamment vers le
début pour présenter l'outil (le wiki dans Moodle) et à la fin pour l'épilogue
(lien avec Wikipedia).

\section{Scénario d'apprentissage}
Les élèves seront amenés à travailler entre 20 et 30 minutes chaque semaine,
pendant 6 semaines, sur l'approfondissement d'un sujet effleuré pendant la
partie précédente du cours (encodage d'un type de donnée pas encore vu en détail
ou étude d'un aspect matériel d'un ordinateur). Le travail s'effectue en grande
partie autour de la conception d'un wiki par chaque groupe. Une fois ce travail
achevé, une transposition est effectuée du mode de travail employé par les
élèves vers la manière dont fonctionne Wikipedia, en clarifiant que les élèves
sont maintenant habilités à y contribuer s'ils le souhaitent.

Voici le calendrier général de la séquence.
\begin{center}
\begin{tabular}{rp{.7\textwidth}}
  semaine 1& les groupes et les sujets sont définis\\
  semaine 2& le wiki du groupe a au moins 2 pages différentes et 3 liens
              externes vers des sources;
              la répartition des tâches au sein du groupe (qui fait quoi) est claire et documentée\\
  semaine 3& toutes les informations utiles ont été trouvées et répertoriées
              (liens vers sources);
              un élève d'un autre groupe devrait pouvoir bien comprendre le sujet grâce à ce wiki\\
  semaine 4& chaque élève a lu le wiki d'un autre groupe et fourni des commentaires utiles à celui-ci\\
  semaine 5& le wiki est fini et complet, en prenant en compte les remarques des
             camarades d'autres groupes qui l'ont lu\\
  semaine 6& présentation orale (10 min) avec support à choix (le wiki lui-même
             ou autre)\\
  semaine 7& debriefing et présentation de Wikipedia
\end{tabular}  
\end{center}

Nous pouvons découper le scénario d'apprentissage en cinq étapes, en suivant le
modèle de Robert Bibeau \autocite{bibeau}.

\subsection{Mise en situation}
L'enseignant explique à la classe que tout en démarrant un nouveau thème de
travail, le thème précédent sera approfondi par des travaux de groupe auxquelles
20 à 30 minutes de temps du cours sera consacré pendant les 6 prochains cours.
La première semaine, il est juste question de constituer des groupes, choisir un
sujet et effectuer une brève recherche préliminaire de sources. Il n'est pas
encore explicitement question de wiki (éviter la surcharge cognitive: pour
l'instant, on cherche un sujet et on essaye de constituer des groupes d'élèves
pouvant fonctionner ensemble de manière efficace et agréable).

Les groupes d'élèves sont constitués et un sujet choisi par chacun (cf. Annexe
1). Les élèves sont invités à choisir un élément de la partie du cours précédent
qu'ils souhaitent approfondir. Le rôle de l'enseignant à cette étape est de se
balader entre les groupes et s'assurer que chacun arrive à choisir un thème qui
est pertinent pour le cours et intéresse réellement les élèves du groupe.

\subsection{Situation d'apprentissage}
Les élèves ont accès assez vite à une version en-ligne du calendrier suivant des
objectifs minimaux à atteindre chaque semaine pour les semaines 2 à 6. Ce
calendrier peut être adapté suivant les éventuels imprévus.

Une fois les groupes constituées, un moment ``frontal'' d'une dizaine de minutes
est utilisé par l'enseignant pour montrer aux élèves les rudiments du
fonctionnement du wiki sur Moodle (notamment comment créer un lien vers une
autre page du wiki, déjà existante ou pas, et comment créer un lien vers une
\textsc{url} externe).

Les élèves sont invités à travailler 20 à 30 minutes chaque semaine sur leur
projet, avec des objectifs minimaux chaque semaine. Aucun ``devoir à la maison''
explicite n'est donné, mais il est indiqué que s'ils le souhaitent, ils peuvent
avancer sur leur projet en dehors des heures de cours.

\subsection{Objectivation}
Chaque élève est amené à évaluer le wiki d'un autre groupe lors de la semaine
4. Cette étape devrait lui permettre de se détacher de son propre travail,
prendre du recul et y revenir avec un regard neuf. De plus, cette étape permet
une évaluation formative par les pairs. Chaque élève se voit assigner le wiki
d'un autre projet lors de la semaine 4. Il reçoit à cet effet une feuille dont
le modèle se trouve à l'Annexe 2. Il devra laisser au moins trois commentaires
constructifs dans le wiki qui lui est assigné, et remplir et rendre la feuille
qui lui a été fournie, en montrant notamment en quoi la navigation d'un autre
wiki lui aura donné des idées pour son propre projet.

\subsection{Situation d'évaluation}
L'évaluation notée (donc ``sommative'') de cette séquence prend en compte trois
groupes de critères distincts:
\begin{itemize}
\item l'atteinte des objectifs minimaux à la fin de chaque semaine (sorte de
  ``contrôle continu'', évaluant la motivation et l'autonomie du groupe)
\item le wiki ``finalisé'' (entre guillemets parce qu'il peut encore être édité
  après: c'est bien un des buts de travailler avec un wiki!)
\item la présentation orale (semaine 6)
\end{itemize}
Le détail des critères se trouve dans l'Annexe 3. Il est modelé partiellement
sur les critères d'évaluation officiels du Travail de Maturité. D'ailleurs, le
mode de travail et d'évaluation devrait permettre aussi de donner aux élèves de
1re année un avant-goût d'un TM de type ``travail de recherche'' dans un domaine
scientifique (ce qui peut être explicité à la fin).

\subsection{Situation de réinvestissement}
En tant qu'épilogue à la séquence, une discussion collective est lancée
concernant l'encyclopédie en-ligne Wikipedia: pourquoi ce nom? est-ce bien un
wiki comme ceux que les élèves ont investi pendant 5 semaines? qui peut
l'éditer? qui peut y écrire des commentaires?

Une activité sur Wikipedia pourra être lancée à ce moment, en fonction du temps
et des choix pédagogiques de l'enseignant.  On pourrait imaginer de lancer les
élèves à la chasse aux informations incorrectes ou manquantes dans des articles
de Wikipedia dont le contenu fait partie du domaine d'expertise de l'élève
chasseur. Un des domaines d'expertise pourrait être la connaissance de leur
établissement scolaire, dont l'article Wikipedia pourrait être améliorable voire
inexistant.  Dans un deuxième temps, une modification des articles pourra être
faite, soit de manière centralisée à travers l'enseignant, soit de nouveau par
petits groupe, sous la forme de mini projets ``amélioration de Wikipedia''.

Conclusion prévue quelle que soit la teneur de l'``épilogue'' à cette séquence:
les élèves pourront par la suite, en tant que ``citoyens numériques'',
contribuer à Wikipedia s'ils le souhaitent. Ils auront eu une première
expérience de travail avec plusieurs wikis et seront donc compétents pour le
faire.

\section{Courant et stratégie pédagogiques}
Une bonne part de la séquence proposée est appelée à se dérouler sous forme de
``petites équipes pour permettre l'apprentissage par les pairs et la
collaboration'', ce qui d'après Kozanitis \cite{kozanitis} correspond au modèle
socio-constructiviste. Kozanitis rappelle par ailleurs que dans ce modèle
didactique, ``l'enseignant doit veiller en permanence les [sic] productions de
l'étudiant et ses processus d'apprentissage''.
% "veiller les productions" ? <-- (sic) [cet article a plein de fautes de français)
Quant aux niveaux des connaissances abordées, ils évoluent sur deux plans
parallèles.

\subsection{Plan des sujets explicites: encodage, etc.}
Dans le plan des connaissances ``externes'' étudiées puis transmises (sujet des
projets de chaque groupe), des connaissances doivent être acquises, mémorisées,
puis analysées et comprises par les élèves pour être finalement reformulées et
restituées à leurs pairs. Si on ajoute la connaissance procédurale consistant en
la mise en place d'exemples explicatifs de leur sujet de recherche (par exemple,
encodage et décodage d'un nombre à virgule flottante) et la connaissance
meta-cognitive (évaluation de leur sujet d'apprentissage afin d'en transmettre
les avantages et inconvénients), on peut constater qu'on traverse tous les
niveaux taxonomiques de Bloom \cite{bloom}, ce qui rend la méthode de cette séquence
particulièrement intéressante.

\subsection{Plan des sujets implicites: Wikis -> Wikipedia -> Open Source}
Dans le plan de l'outil utilisé pour collaborer et transmettre l'information,
l'outil employé pour gérer l'acquisition mais surtout le partage et la
retransmission des connaissances (le wiki) devient à son tour une source de
connaissance.  Pour ce deuxième but implicite, la navigation dans les niveaux est
bien différente: On commence par les connaissances procédurales avec
l'utilisation des wikis de groupe puis on s'attaque aux autres niveaux de
connaissance: premièrement basiques avec l'apport de l'enseignant qui présente
Wikipedia puis de haut niveau avec les retours ``philosophiques'' sur
l'importance de Wikipedia et, plus largement, des courants \textit{open source},
\textit{logiciel libre}, etc.

\section{Difficultés attendues et problèmes potentiels}
On peut s'attendre à rencontrer plusieurs difficultés dans le cadre de la
réalisation de cette séquence pédagogique.

\subsection{Attitude des élèves}
Il faut espérer que les élèves soient demandeurs, ce qui dépend de nombreux
facteurs comme la capacité à comprendre les notions et à en percevoir
l'intérêt. Si les sujets sont trop complexes (voir ci-dessous), les élèves
risquent de n'y rien comprendre et de rentrer dans la séquence en mode
``minimaliste''; on peut en effet s'attendre à ce que certains élèves se
contentent de copier-coller les contenus et ne retiennent pas grand-chose des
sujets explicites de recherche.

Toutefois, cette difficulté n'est pas imputable à la modalité de cette séquence.
L'autre sujet (implicite) peut aussi intéresser les élèves\ldots ou pas! Il est
probable que l'aspect de partage, d'utopie internet, voire scientifique ne
rentre pas dans la plus haute priorité intellectuelle de l'élève. A nous de
trouver les accroches en taquinant leurs conceptions pour réveiller l'intérêt
des élèves trop conformistes ou ``scolaires'' (dans le mauvais sens du terme).

\subsection{Complexité des connaissances}
La complexité des sujets doit être gérée par l'enseignant. Celui-ci doit veiller
à cadrer les sujets pour qu'ils restent dans un intervalle de complexité
accessible à l'ensemble des élèves. Sinon, les sujets d'étude risquent de
prendre le dessus sur l'outil (wiki) qu'on veut mettre en lumière. En effet, le
temps-cerveau nécessaire à la compréhension des sujets explicites va empêcher
les élèves d'approfondir la notion de wiki et Wikipedia.  La difficulté
intrinsèque de l'apprentissage des wikis ne semble pas, elle, inatteignable.

\subsection{Suivi des projets}
De nombreux problèmes organisationnels sont à prévoir pour cadrer l'avancement
des groupes, s'assurer que certains ne restent pas bloqués, ne pas donner toute
l'attention aux groupes ambitieux ou paniqués qui peuvent solliciter toute
l'attention de l'enseignant au détriment des autres.  Une part d'improvisation
sera certainement nécessaire pour suivre de manière efficace et bienveillante
les projets.

\section{Motivation des élèves et valeurs personnelles}
Comme l'indique Duplessis \cite{duplessis}, nous pouvons nous attendre à des
\textit{à priori} des élèves ``sur le monde en général et les objets d'étude en
particulier''. En ce qui nous concerne, nous nous attendons à ce qu'ils
s'imaginent que le savoir appris à l'école en général et sur Wikipedia en
particulier sont des sources de connaissance dont les élèves ne sont ni auteurs ni
acteurs. Duplessis nous indique qu'``apprendre nécessite que ces conceptions
soient progressivement transformées''. Nous espérons qu'à travers le scénario
d'apprentissage proposé, l'élève devienne plus ``acteur'' de son acquisition de
connaissance, et que le futur citoyen aie conscience qu'il peut aussi contribuer
à la connaissance générale au moyen de media collaboratifs comme Wikipedia.
De plus, un ``objectif caché'' lié à nos valeurs personnelles est de faire
découvrir la force du logiciel libre et du \textit{open content} à nos élèves,
sans en faire pour autant un exposé théorique.
 
Nous espérons aussi que le mode de travail sera motivant pour les élèves. Viau
\cite{viau} argumente que l' ``apprentissage par projet est l'activité
d'apprentissage qui est perçue par les étudiants comme la plus utile, dans
laquelle ils se sentent le plus compétents et sur laquelle ils ont le sentiment
d'avoir le plus de contrôle''. D'après Rolland Viau, ces sentiments de
\textit{compétence} et \textit{contrôle} sont fondamentaux pour que l'élève se
sente bien dans son rôle et puisse ainsi être efficace dans son apprentissage.
En l'occurrence, le contrôle est justement intrinsèquement lié au travail avec
des outils \textit{ouverts}, tels que l'encyclopédie en-ligne Wikipedia: on
n'est pas juste utilisateur, on peut apporter sa contribution à l'outil.

L'apprentissage de la Science Informatique devrait aller de pair avec un
changement du regard sur le monde. Jean-Pierre Astolfi \cite{astolfi} considère
justement qu'une discipline ``ne se définit pas d'abord par les objets de son
domaine empirique, mais par le type de cadrage théorique original que produisent
les concepts'' et qu'elle ``n'est pas une accumulation de données [\ldots] mais
l'entrée dans une interprétation experte du monde, plus puissante que celle du
sens commun''.

À travers cette séquence pédagogique, l'élève aura pu affleurer par la pratique
le mode de création des connaissances scientifiques par confrontation et
le partage; c'est ce mode de fonctionnement qui a permis aux sciences
expérimentales d'avancer de manière efficace et de produire de nouveaux objets
abstraits ou concrets qui ont révolutionné notre quotidien voire même l'essence
de l'Humain (électricité, informatique, internet, etc.). L'élève pourra en
dégager un facteur d'explosion exponentielle des techniques et des savoirs
puisque les nouveaux objets produits par la science permettent d'accélérer la
production des savoirs ultérieurs. Ainsi, le rôle de l'internet, notamment grâce
à Wikipedia permet un accès à un savoir encyclopédique de qualité. On n'y trouve
pas forcément les avancées actuelles des recherches car son but est de ne
refléter que les connaissances établies mais on peut imaginer que son rôle
change ou que d'autres outils plus adaptés à cette exploration intellectuelle
fondamentale apparaissent et nos élèves pourraient bien en être les futurs
auteurs\ldots

\pagebreak
% On finit avec sa bibliographie, son colophon et ses annexes
\printbibliography  % pas numéroté, automatiquement au toc
\vfill
% ça c'est un colophon:
% C'est surtout du prosélitisme!
% ben oui, pourquoi pas?
\emph{Document réalisé en \href{https://www.latex-project.org/}{\LaTeX}.
  Bibliographie gérée avec \href{http://www.bibtex.org/}{Bib\TeX}}.\par
\emph{Sources disponibles \href{https://github.com/Dalker/didac_010}{sur github}.}

% et puis on inclut des annexes
\section*{Annexe 1: définition des groupes et du projet} % pas numéroté, pas au toc
\addcontentsline{toc}{section}{Annexes: supports de cours}  % ajouter au toc
                                % (une entrée pour tous les annexes)
\includegraphics[width=.95\textwidth]{annexes/projet1.pdf}

\section*{Annexe 2: évaluation par les pairs} % pas numéroté, pas au toc
\includegraphics[width=.95\textwidth]{annexes/pairs.pdf}

\section*{Annexe 3: grille d'évaluation du projet} % pas numéroté, pas au toc
\includegraphics[width=.95\textwidth]{annexes/evaluation.pdf}


\end{document}