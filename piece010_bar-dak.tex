\documentclass[11pt,bibliography=totoc]{scrartcl}
% \usepackage[nottoc,numbib]{tocbibind}
\bibliographystyle{plain}
\author{Juan-Carlos Barros et Daniel Kessler}
\date{\today}
\title{Conception d'une séquence d'enseignement\\\medskip
  \large GymInf, Didactique de l'informatique, Rendu 010}

%% CONFIGURATION
% configuration générale
\usepackage[l2tabu, orthodox]{nag}  % demander indication d'usages désuets
\usepackage{microtype}  % subtiles améliorations de certains défauts
\usepackage[french]{babel}  % langue du document

% configuration des polices de caractère / aspect du titre
\usepackage[utf8]{inputenc}  % nécessaire en MS-Windows, pas dans Linux
                             % (probablement pas dans MacOS non plus)

% Possibles changements de police ci-dessous (en décomenter un, ou pas)
% \usepackage{tgtermes} %  (semble nécessiter lualatex)
% \usepackage[T1]{fontenc}

% pdfisation avec liens
\usepackage[pdfstartview=FitH,
pdfauthor={Juan-Carlos Barros et Daniel Kessler},
pdftitle={Conception d'une séquence}]{hyperref}
\usepackage{xcolor}
\definecolor{pdflinkcolor}{RGB}{30,60, 120}
\hypersetup{colorlinks=true, allcolors=pdflinkcolor}

\begin{document}

\maketitle
\tableofcontents

\pagebreak

\section{Contexte institutionnel et disciplinaire}
Nous enseignons tous deux au niveau gymnasial dans le Collège de Genève. Dans ce
cadred, nous avons cette année plusieurs groupes d'élèves de 1ère année
informatique discipline fondamentale, auxquels s'adresse le travail présenté
dans ce document. Ces groupes sont constitués de 16 élèves, issus
potentiellement de plusieurs ``classes'' (le concept de ``classe'' est très flou
au Collège, les élèves changeant souvent de groupe pour des matières à effectif
réduit, selon leur niveau de maths, leurs choix d'options, etc.).

La séquence d'enseignement que nous allons présenter possède deux
caractéristiques distinctives:
\begin{itemize}
\item Elle se déroule sur plusieurs leçons, en prenant 20-30 minutes du cours à
  chaque fois.
\item Elle combine des objectifs explicites pour les élèves avec des objectifs
  implicites qui seront explicités plus tard.
\end{itemize}

Cette séquence s'insère en deuxième partie du premier semestre, après avoir
travaillé sur le code binaire, vu quelques encodages simples (nombres entiers,
caractères \textsc{ascii}, bitmaps) et travaillé un petit peu sur les composants
matériels d'un ordinateur, éléments du chapitre \textit{Information et Données}
du plan d'étude cantonal genevois \cite{pecinfo}.

L'objectif explicite initial de la séquence présentée dans ce document est
d'approfondir un des éléments précédents, à travers un travail de groupe sur un
sujet qui n'a été qu'effleuré pendant la première partie de l'année scolaire
(par exemple l'encodage des nombres à virgule flottante, la parenté entre les
composants d'un smartphone et ceux d'un PC, etc.)

Le deuxième objectif, initialement implicite, est de comprendre comment
fonctionne un \textit{wiki} en en utilisant un comme outil, pour découvrir
finalement que certains wikis sont très ouverts aux contributions publiques,
notamment \textit{Wikipedia}. Après avoir expérimenté le travail sur un wiki, un
élève devrait comprendre le mécanisme de base et se sentir capable de contribuer
à Wikipedia s'il ou elle y repère une coquille à corriger, une information
caduque ou incomplète, etc. Ce deuxième objectif participe à plusieurs thèmes du
plan d'étude notamment les \textit{Réseaux}, \textit{Information et Données} et
la \textit{Citoyenneté Numérique}.

\section{Objectifs pédagogiques}
La majorité de cette séquence se déroule en suivant le paradigme
\textit{socio-constructiviste} [citer un article ici]. En effet, les élèves
contruisent une partie de leur propre connaissance en effectuant leurs propres
recherches et en collaborant avec leurs camarades.

bla bla en plus, notamment sur le rôle du prof

Cependant, cela n'empêche pas quelques moments ``frontaux'', notamment bla bla 

\section{Scénario d'apprentissage}
D'abord\ldots

(voir Annexe machin)

Ensuite\ldots

Notamment\ldots

(voir Annexe truc)

\section{Courant et stratégie pédagogiques}

\section{Difficultés attendues et problèmes potentiels}

\section{Aspects psychologiques et organisationnels}

\bibliography{biblio}

\addcontentsline{toc}{section}{Annexes: supports de cours}

\vfill
\emph{Document réalisé en \href{https://www.latex-project.org/}{\LaTeX}}
\end{document}